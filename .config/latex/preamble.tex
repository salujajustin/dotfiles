%%   Preamble Command Cheat Sheet 
%   \usepackage[options]{package name}: Include packages to enhance document design and functionality  
%   \newcommand{cmd}[args]{def}: Define a new command  
%   \ensuremath{<math commands>}: Allows mathematical typesetting in math and text modes
%   \mathbb{<upper case letter>}: Blackboard bold typefaces
%   \let<new-command>[[<spaces>]=]<original-command>: Allows you to copy the content of a command into a new command
%   \definecolor{name}{model}{color-spec}: Define your own custom color

%% Essential packages
\usepackage[utf8]{inputenc}    % Standard input encoding
\usepackage[T1]{fontenc}       % Standard font encoding 
\usepackage{textcomp}          % Text companion fonts e.g. Bullet, copyright
\usepackage[english]{babel}    % Multilingual support i.e. [french]
\usepackage{url}			   % URL typesetting via \url
\usepackage{enumitem}          % Control layout of itemize, enumerate, description

%% Graphics, Tables and Colors
\usepackage{graphicx} 		   % Accommodate need for inclusion of graphics 
\usepackage{float}  		   % Improved interface for floating objects 
\usepackage{booktabs}          % Publication quality tables
\usepackage{pdfrender}

%% Appearance and Text
\usepackage{parskip}           % Don't indent paragraphs, leave some space between them
\usepackage{emptypage} 		   % Hide page number and headings from appearing when page is empty  
\usepackage{subcaption}        % Support for sub-captions on figures 
\usepackage{multicol}          % Allows switching between one and mulicolumn format on the same page 
\usepackage{xcolor}            % Access to color tints, shades, tones, and mixes of arbitrary colors

%% Fonts: Uncomment to use
% \usepackage{cmbright}          % Computer Modern Bright fonts

%% Code 
\usepackage{pythonhighlight}   % Python syntax highlighting

%% Math
\usepackage{amsmath, amsfonts, mathtools, amsthm, amssymb}
\usepackage{pifont}            % Additional symbols
\usepackage{mathrsfs}		   % A calligraphic font for fancy capital letters i.e. \mathscr{ABC}
\usepackage{cancel}            % Place lines through math formulae
\usepackage{bm}                % Bold math
\usepackage{systeme}           % Easily typeset systems of equations (French package)
\usepackage{stmaryrd}          % Symbols for theoretical computer science, i.e. \lightning
\usepackage{siunitx}		   % (SI) units package

% Shortcuts for number sets 
\newcommand\N{\ensuremath{\mathbb{N}}}
\newcommand\R{\ensuremath{\mathbb{R}}}
\newcommand\Z{\ensuremath{\mathbb{Z}}}
\renewcommand\O{\ensuremath{\emptyset}}
\newcommand\Q{\ensuremath{\mathbb{Q}}}
\newcommand\C{\ensuremath{\mathbb{C}}}
% Shortcut to denote contradiction
\newcommand\contra{\scalebox{1.5}{$\lightning$}}
% Put x \to \infty below \lim
\let\svlim\lim\def\lim{\svlim\limits}
% Make implies and impliedby shorter
\let\implies\Rightarrow
\let\impliedby\Leftarrow
\let\iff\Leftrightarrow
% Replace variant definitions of phi and epsilon 
% \let\epsilon\varepsilon
% \let\phi\varphi

%% Utilities
% Command for short corrections 
% Usage: 1+1=\correct{3}{2}
\definecolor{correct}{HTML}{009900}  % Green color
\newcommand\correct[2]{\ensuremath{\:}{\color{red}{#1}}\ensuremath{\to }{\color{correct}{#2}}\ensuremath{\:}}
\newcommand\green[1]{{\color{correct}{#1}}}
% Insert a horizontal line break
\newcommand\hr{                
    \noindent\rule[0.5ex]{\linewidth}{0.5pt}
} 
% Hide notes from the pdf file: \hide{<text>} 
\newcommand\hide[1]{}          


% Figure support as explained in my blog post.
\usepackage{import}
\usepackage{xifthen}
\pdfminorversion=7
\usepackage{pdfpages}
\usepackage{transparent}
\newcommand{\incfig}[1]{%
	\def\svgwidth{\columnwidth}
	\import{./figures/}{#1.pdf_tex}
}

%% Environments
\makeatother  % 
% For box around Definition, Theorem, \ldots
\usepackage{mdframed}
\mdfsetup{skipabove=1.0em,skipbelow=0.0em}
\theoremstyle{definition}
\newmdtheoremenv[nobreak=true]{definition}{Definition}
\newmdtheoremenv[nobreak=true]{prop}{Proposition}
\newmdtheoremenv[nobreak=true]{theorem}{Theorem}
\newmdtheoremenv[nobreak=true]{corollary}{Corollary}
\newtheorem*{example}{Example}
\newtheorem*{notation}{Notation}
\newtheorem*{previouslyseen}{As previously seen}
\newtheorem*{remark}{Remark}
\newtheorem*{note}{Note}
\newtheorem*{problem}{Problem}
\newtheorem*{observe}{Observe}
\newtheorem*{property}{Property}
\newtheorem*{intuition}{Intuition}





% End example and intermezzo environments with a small diamond (just like proof
% environments end with a small square)
\usepackage{etoolbox}
\AtEndEnvironment{vb}{\null\hfill$\diamond$}%
\AtEndEnvironment{intermezzo}{\null\hfill$\diamond$}%
% \AtEndEnvironment{opmerking}{\null\hfill$\diamond$}%

% Fix some spacing
% http://tex.stack exchange.com/questions/22119/how-can-i-change-the-spacing-before-theorems-with-asthma
\makeatletter
\def\thm@space@setup{%
  \thm@preskip=\parskip \thm@postskip=0pt
}

% Green check mark
\newcommand*{\greencheck}{%
  \textpdfrender{
    TextRenderingMode=FillStroke,
    LineWidth=.5pt, % half of the line width is outside the normal glyph
}{{\color{green}\checkmark}}%
}
% Red X
\newcommand{\redx}{{\color{red}\ding{55}}}


% Exercise 
% Usage:
% \oefening{5}
% \suboefening{1}
% \suboefening{2}
% \suboefening{3}
% gives
% Oefening 5
%   Oefening 5.1
%   Oefening 5.2
%   Oefening 5.3
\newcommand{\oefening}[1]{%
    \def\@oefening{#1}%
    \subsection*{Oefening #1}
}
\newcommand{\suboefening}[1]{%
    \subsubsection*{Oefening \@oefening.#1}
}


% \lecture starts a new lecture 
%
% Usage:
% \lecture{1}{di 12 feb 2019 16:00}{Inleiding}
%
% This adds a section heading with the number / title of the lecture and a
% margin paragraph with the date.

% I use \dateparts here to hide the year (2019). This way, I can easily parse
% the date of each lecture unambiguously while still having a human-friendly
% short format printed to the pdf.

\usepackage{xifthen}
\def\testdateparts#1{\dateparts#1\relax}
\def\dateparts#1 #2 #3 #4 #5\relax{
    \marginpar{\small\textsf{\mbox{#1 #2 #3 #5}}}
}

\def\@lecture{}%
\newcommand{\lecture}[3]{
    \ifthenelse{\isempty{#3}}{%
        \def\@lecture{Lecture #1}%
    }{%
        \def\@lecture{Lecture #1: #3}%
    }%
    \subsection*{\@lecture}
    \marginpar{\small\textsf{\mbox{#2}}}
}



% These are the fancy headers
\usepackage{fancyhdr}
\pagestyle{fancy}

% LE: left even
% RO: right odd
% CE, CO: center even, center odd
% My name for when I print my lecture notes to use for an open book exam.
% \fancyhead[LE,RO]{Gilles Castel}

% \fancyhead[RO,LE]{\@lecture} % Right odd,  Left even
% \fancyhead[RE,LO]{}          % Right even, Left odd

% \fancyfoot[RO,LE]{\thepage}  % Right odd,  Left even
% \fancyfoot[RE,LO]{}          % Right even, Left odd
% \fancyfoot[C]{\leftmark}     % Center

\makeatother



%% Editing PDFs 
% Todonotes and inline notes in fancy boxes
\usepackage{todonotes}		% Insert todo items e.g. missing figures, list all items 
\usepackage{tcolorbox}		% Framed and colored boxes environment

% Make boxes breakable
\tcbuselibrary{breakable}

% Correction 
% Usage: 
% \begin{correction}
%     Lorem ipsum dolor sit amet, consetetur sadipscing elitr, sed diam nonumy eirmod
%     tempor invidunt ut labore et dolore magna aliquyam erat, sed diam voluptua. At
%     vero eos et accusam et justo duo dolores et ea rebum. Stet clita kasd gubergren,
%     no sea takimata sanctus est Lorem ipsum dolor sit amet.
% \end{correction}
\newenvironment{correction}{\begin{tcolorbox}[
    arc=0mm,
    colback=white,
    colframe=green!60!black,
    title=Solution,
    fonttitle=\sffamily,
    breakable
]}{\end{tcolorbox}}

% Noot is note in Dutch. Same as 'verbetering' but color of box is different
\newenvironment{noot}{\begin{tcolorbox}[
    arc=0mm,
    colback=white,
    colframe=white!60!black,
    title=Output,
    fonttitle=\sffamily,
    breakable
]}{\end{tcolorbox}}





% Fix issue when multiple pdf pages are included in a single page
\pdfsuppresswarningpagegroup=1

% Name 
\author{Justin Saluja}
